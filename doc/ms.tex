% Every Latex document starts with a documentclass command
\documentclass[a4paper, 11pt]{article}

% Load some packages
\usepackage{graphicx} % This allows you to put figures in
\usepackage{natbib}   % This allows for relatively pain-free reference lists
\usepackage[left=3cm,top=3cm,right=3cm]{geometry} % The way I like the margins

% This helps with figure placement
\renewcommand{\topfraction}{0.85}
\renewcommand{\textfraction}{0.1}
\parindent=0cm

% Set values so you can have a title
\title{New Paper}
\author{Me}
\date{\today}

% Document starts here
\begin{document}

% Actually put the title in
\maketitle

\abstract{This is the abstract}

\section{Introduction}
The $x$-values from runs 1 and 2 are:
\begin{eqnarray}
\{x_{11}, x_{12}, ..., x_{1n}\}\\
\{x_{21}, x_{22}, ..., x_{2n}\}
\end{eqnarray}

The prior for what happened in run 1 is:
\begin{eqnarray}
x_{11} \sim \textnormal{Uniform}(0, 1)\\
x_{12} | x_{11} \sim \textnormal{Uniform}(0, x_{11})\\
x_{13} | x_{11}, x_{12} \sim \textnormal{Uniform}(0, x_{12})
\end{eqnarray}

\end{document}

